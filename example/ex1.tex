\documentclass[11pt]{jsarticle}
\begin{document}

\title{情報科学レポート}
\author{山口勇希}
\maketitle

\section{はじめに}

この文書の先頭にはタイトル,著者名,日付が出力されています。
特定の日付を指定することもできます。

\section{見出し}

そして,セクションの見出しが出力されています。
セクションの番号は自動的に付きます。

\section{箇条書き}

これは番号を振らない箇条書きです。

\begin{itemize}
  \item ちゃお
  \item りぼん
  \item なかよし
\end{itemize}

これは番号を振る箇条書きです。

\begin{enumerate}
  \item 富士
  \item 鷹
  \item なすび
\end{enumerate}

これは語句説明の箇条書きです。

\begin{description}
  \item[遷移クリープ] クリープ速度が次第に減少
  \item[定常クリープ] クリープ速度がほとんど時間的に変化しない
  \item[加速クリープ] 破断するまで連続的にクリープ速度が増加
\end{description}

\section{改行}

ほげほげ

ここは段落改行 \par
ここも段落改行 \\
ここは強制改行

\noindent
ここも強制改行

\section{おわりに}

これは一段組の例ですが,二段組に変更することもできます。

\end{document}
